\begin{pages}
    \begin{Rightside}
    \selectlanguage{greek}
        \beginnumbering
		\pstart[
				\chapter{Ἐν ἀρχῇ ἦν ὁ λόγος}
				\markboth{In the beginning was the Word}
				]
		\renewcommand{\LettrineFontHook}{\PHtitl}
		\lettrine[lines=3]{Ἐ}{ν} ἀρχῇ ἦν ὁ Λόγος, καὶ ὁ Λόγος ἦν πρὸς τὸν Θεόν, καὶ Θεὸς ἦν ὁ Λόγος. Οὗτος ἦν ἐν ἀρχῇ πρὸς τὸν Θεόν. πάντα δι’ αὐτοῦ ἐγένετο, καὶ χωρὶς αὐτοῦ ἐγένετο οὐδὲ ἕν ὃ γέγονεν. ἐν αὐτῷ ζωὴ ἦν, καὶ ἡ ζωὴ ἦν τὸ φῶς τῶν ἀνθρώπων. καὶ τὸ φῶς ἐν τῇ σκοτίᾳ φαίνει, καὶ ἡ σκοτία αὐτὸ οὐ κατέλαβεν. 
		\pend
		\pstart		
		Ἐγένετο ἄνθρωπος, ἀπεσταλμένος παρὰ Θεοῦ, ὄνομα αὐτῷ Ἰωάνης· οὗτος ἦλθεν εἰς μαρτυρίαν, ἵνα μαρτυρήσῃ περὶ τοῦ φωτός, ἵνα πάντες πιστεύσωσιν δι’ αὐτοῦ. οὐκ ἦν ἐκεῖνος τὸ φῶς, ἀλλ’ ἵνα μαρτυρήσῃ περὶ τοῦ φωτός. Ἦν τὸ φῶς τὸ ἀληθινὸν, ὃ φωτίζει πάντα ἄνθρωπον, ἐρχόμενον εἰς τὸν κόσμον. ἐν τῷ κόσμῳ ἦν, καὶ ὁ κόσμος δι’ αὐτοῦ ἐγένετο, καὶ ὁ κόσμος αὐτὸν οὐκ ἔγνω. εἰς τὰ ἴδια ἦλθεν, καὶ οἱ ἴδιοι αὐτὸν οὐ παρέλαβον. ὅσοι δὲ ἔλαβον αὐτόν, ἔδωκεν αὐτοῖς ἐξουσίαν τέκνα Θεοῦ γενέσθαι, τοῖς πιστεύουσιν εἰς τὸ ὄνομα αὐτοῦ, οἳ οὐκ ἐξ αἱμάτων οὐδὲ ἐκ θελήματος σαρκὸς οὐδὲ ἐκ θελήματος ἀνδρὸς ἀλλ’ ἐκ Θεοῦ ἐγεννήθησαν. 
		\pend
		\pstart
		Καὶ ὁ Λόγος σὰρξ ἐγένετο καὶ ἐσκήνωσεν ἐν ἡμῖν, καὶ ἐθεασάμεθα τὴν δόξαν αὐτοῦ, δόξαν ὡς μονογενοῦς παρὰ Πατρός, πλήρης χάριτος καὶ ἀληθείας. Ἰωάνης μαρτυρεῖ περὶ αὐτοῦ καὶ κέκραγεν λέγων Οὗτος ἦν ὃν εἶπον Ὁ ὀπίσω μου ἐρχόμενος ἔμπροσθέν μου γέγονεν, ὅτι πρῶτός μου ἦν. ὅτι ἐκ τοῦ πληρώματος αὐτοῦ ἡμεῖς πάντες ἐλάβομεν, καὶ χάριν ἀντὶ χάριτος· ὅτι ὁ νόμος διὰ Μωϋσέως ἐδόθη, ἡ χάρις καὶ ἡ ἀλήθεια διὰ Ἰησοῦ Χριστοῦ ἐγένετο. Θεὸν οὐδεὶς ἑώρακεν πώποτε· μονογενὴς Θεὸς ὁ ὢν εἰς τὸν κόλπον τοῦ Πατρὸς, ἐκεῖνος ἐξηγήσατο.
		\pend
		\pstart
		Καὶ αὕτη ἐστὶν ἡ μαρτυρία τοῦ Ἰωάνου ὅτε ἀπέστειλαν πρὸς αὐτὸν οἱ Ἰουδαῖοι ἐξ Ἱεροσολύμων ἱερεῖς καὶ Λευείτας ἵνα ἐρωτήσωσιν αὐτόν Σὺ τίς εἶ; καὶ ὡμολόγησεν καὶ οὐκ ἠρνήσατο, καὶ ὡμολόγησεν ὅτι Ἐγὼ οὐκ εἰμὶ ὁ Χριστός. καὶ ἠρώτησαν αὐτόν Τί οὖν; σὺ Ἡλείας εἶ; καὶ λέγει Οὐκ εἰμί. Ὁ προφήτης εἶ σύ; καὶ ἀπεκρίθη Οὔ. εἶπαν οὖν αὐτῷ Τίς εἶ; ἵνα ἀπόκρισιν δῶμεν τοῖς πέμψασιν ἡμᾶς· τί λέγεις περὶ σεαυτοῦ; ἔφη Ἐγὼ φωνὴ βοῶντος ἐν τῇ ἐρήμῳ Εὐθύνατε τὴν ὁδὸν Κυρίου, καθὼς εἶπεν Ἡσαΐας ὁ προφήτης. 
		\pend
		\pstart
		Καὶ ἀπεσταλμένοι ἦσαν ἐκ τῶν Φαρισαίων. καὶ ἠρώτησαν αὐτὸν καὶ εἶπαν αὐτῷ Τί οὖν βαπτίζεις εἰ σὺ οὐκ εἶ ὁ Χριστὸς οὐδὲ Ἡλείας οὐδὲ ὁ προφήτης; ἀπεκρίθη αὐτοῖς ὁ Ἰωάνης λέγων Ἐγὼ βαπτίζω ἐν ὕδατι· μέσος ὑμῶν στήκει ὃν ὑμεῖς οὐκ οἴδατε, ὁ ὀπίσω μου ἐρχόμενος, οὗ οὐκ εἰμὶ ἐγὼ ἄξιος ἵνα λύσω αὐτοῦ τὸν ἱμάντα τοῦ ὑποδήματος. Ταῦτα ἐν Βηθανίᾳ ἐγένετο πέραν τοῦ Ἰορδάνου, ὅπου ἦν ὁ Ἰωάνης βαπτίζων.
		\pend
		\pstart
		Τῇ ἐπαύριον βλέπει τὸν Ἰησοῦν ἐρχόμενον πρὸς αὐτόν, καὶ λέγει Ἴδε ὁ Ἀμνὸς τοῦ Θεοῦ ὁ αἴρων τὴν ἁμαρτίαν τοῦ κόσμου. οὗτός ἐστιν ὑπὲρ οὗ ἐγὼ εἶπον Ὀπίσω μου ἔρχεται ἀνὴρ ὃς ἔμπροσθέν μου γέγονεν, ὅτι πρῶτός μου ἦν. κἀγὼ οὐκ ᾔδειν αὐτόν, ἀλλ’ ἵνα φανερωθῇ τῷ Ἰσραὴλ, διὰ τοῦτο ἦλθον ἐγὼ ἐν ὕδατι βαπτίζων. Καὶ ἐμαρτύρησεν Ἰωάνης λέγων ὅτι Τεθέαμαι τὸ Πνεῦμα καταβαῖνον ὡς περιστερὰν ἐξ οὐρανοῦ, καὶ ἔμεινεν ἐπ’ αὐτόν. κἀγὼ οὐκ ᾔδειν αὐτόν, ἀλλ’ ὁ πέμψας με βαπτίζειν ἐν ὕδατι ἐκεῖνός μοι εἶπεν Ἐφ’ ὃν ἂν ἴδῃς τὸ Πνεῦμα καταβαῖνον καὶ μένον ἐπ’ αὐτόν, οὗτός ἐστιν ὁ βαπτίζων ἐν Πνεύματι Ἁγίῳ. κἀγὼ ἑώρακα, καὶ μεμαρτύρηκα ὅτι οὗτός ἐστιν ὁ Υἱὸς τοῦ Θεοῦ.
		\pend
		\pstart
		Τῇ ἐπαύριον πάλιν εἱστήκει ὁ Ἰωάνης καὶ ἐκ τῶν μαθητῶν αὐτοῦ δύο, καὶ ἐμβλέψας τῷ Ἰησοῦ περιπατοῦντι λέγει Ἴδε ὁ Ἀμνὸς τοῦ Θεοῦ. καὶ ἤκουσαν οἱ δύο μαθηταὶ αὐτοῦ λαλοῦντος καὶ ἠκολούθησαν τῷ Ἰησοῦ. στραφεὶς δὲ ὁ Ἰησοῦς καὶ θεασάμενος αὐτοὺς ἀκολουθοῦντας λέγει αὐτοῖς Τί ζητεῖτε; οἱ δὲ εἶπαν αὐτῷ Ῥαββεί, (ὃ λέγεται μεθερμηνευόμενον Διδάσκαλε,) ποῦ μένεις; λέγει αὐτοῖς Ἔρχεσθε καὶ ὄψεσθε. ἦλθαν οὖν καὶ εἶδαν ποῦ μένει, καὶ παρ’ αὐτῷ ἔμειναν τὴν ἡμέραν ἐκείνην· ὥρα ἦν ὡς δεκάτη. Ἦν Ἀνδρέας ὁ ἀδελφὸς Σίμωνος Πέτρου εἷς ἐκ τῶν δύο τῶν ἀκουσάντων παρὰ Ἰωάνου καὶ ἀκολουθησάντων αὐτῷ· εὑρίσκει οὗτος πρῶτον τὸν ἀδελφὸν τὸν ἴδιον Σίμωνα καὶ λέγει αὐτῷ Εὑρήκαμεν τὸν Μεσσίαν (ὅ ἐστιν μεθερμηνευόμενον Χριστός). ἤγαγεν αὐτὸν πρὸς τὸν Ἰησοῦν. ἐμβλέψας αὐτῷ ὁ Ἰησοῦς εἶπεν Σὺ εἶ Σίμων ὁ υἱὸς Ἰωάνου, σὺ κληθήσῃ Κηφᾶς (ὃ ἑρμηνεύεται Πέτρος). 
		\pend
		\pstart
		Τῇ ἐπαύριον ἠθέλησεν ἐξελθεῖν εἰς τὴν Γαλιλαίαν. καὶ εὑρίσκει Φίλιππον. καὶ λέγει αὐτῷ ὁ Ἰησοῦς Ἀκολούθει μοι. ἦν δὲ ὁ Φίλιππος ἀπὸ Βηθσαϊδά, ἐκ τῆς πόλεως Ἀνδρέου καὶ Πέτρου. εὑρίσκει Φίλιππος τὸν Ναθαναὴλ καὶ λέγει αὐτῷ Ὃν ἔγραψεν Μωϋσῆς ἐν τῷ νόμῳ καὶ οἱ προφῆται εὑρήκαμεν, Ἰησοῦν υἱὸν τοῦ Ἰωσὴφ τὸν ἀπὸ Ναζαρέτ. καὶ εἶπεν αὐτῷ Ναθαναήλ Ἐκ Ναζαρὲτ δύναταί τι ἀγαθὸν εἶναι; λέγει αὐτῷ ὁ Φίλιππος Ἔρχου καὶ ἴδε. εἶδεν Ἰησοῦς τὸν Ναθαναὴλ ἐρχόμενον πρὸς αὐτὸν καὶ λέγει περὶ αὐτοῦ Ἴδε ἀληθῶς Ἰσραηλείτης, ἐν ᾧ δόλος οὐκ ἔστιν. λέγει αὐτῷ Ναθαναήλ Πόθεν με γινώσκεις; ἀπεκρίθη Ἰησοῦς καὶ εἶπεν αὐτῷ Πρὸ τοῦ σε Φίλιππον φωνῆσαι ὄντα ὑπὸ τὴν συκῆν εἶδόν σε. ἀπεκρίθη αὐτῷ Ναθαναήλ Ῥαββεί, σὺ εἶ ὁ Υἱὸς τοῦ Θεοῦ, σὺ Βασιλεὺς εἶ τοῦ Ἰσραήλ. ἀπεκρίθη Ἰησοῦς καὶ εἶπεν αὐτῷ Ὅτι εἶπόν σοι ὅτι εἶδόν σε ὑποκάτω τῆς συκῆς, πιστεύεις; μείζω τούτων ὄψῃ. καὶ λέγει αὐτῷ Ἀμὴν ἀμὴν λέγω ὑμῖν, ὄψεσθε τὸν οὐρανὸν ἀνεῳγότα καὶ τοὺς ἀγγέλους τοῦ Θεοῦ ἀναβαίνοντας καὶ καταβαίνοντας ἐπὶ τὸν Υἱὸν τοῦ ἀνθρώπου.		
		\pend
        \endnumbering
    \end{Rightside}
    \begin{Leftside}
        \beginnumbering
        \pstart[
        			\chapter{In the Beginning Was the Word}
        			]
        		\renewcommand\LettrineFontHook{\Zallmanfamily}
		\lettrine[lines=3]{I}{n} the beginning was the Word and the Word was with God and God was the Word; this was (that which was) in the beginning with God. Everything happened through (because of) Him and without Him, none of the things that happened (would have actually) happened. Within Him was life and (the) life was the light of mankind; and the light shines in the darkness, yet (lit. and) the darkness did not understand (grasp) it. 
		\pend
		\pstart
		There was a man sent from God whose name was John; he went into witness so that he may bear witness of the light (and so) that everyone may believe through (because of) him. He (himself) was not the light, but (he was there simply) to bear witness of the light. That (or he?) was the true light, which lights every person coming into the world (cosmos). He was in the world — and the world world was (happened) because of him (was created by him?) —, yet the world did not know him. He went to his own (kind?), yet his own did not take (welcome?) him. Whosever did take (welcome?) him, however, to them he gave the authority to become children of God; (namely) those who believe in his name. Those which were born not of blood or of the will of the flesh or of the will of man, but of God. 
		\pend
		\pstart
		And the Word became (as) flesh and lived within us; and we saw his glory — the glory like the Father’s only-born (child), filled with joy and truth. John bore witness of him and shouted saying, “This is he about whom I said, ‘(He) who walks behind me — (even though) he came before me; for he was before me.’” For we have all received (taken) from his fullness and (have received?) grace in return for (instead of, for) grace. Because the law was given by Moses, grace and truth came (happened) because of Jesus Christ. Nobody has ever seen God. The only God; the one who is in the Father’s bosom — he declared him. 
		\pend
		\pstart
		And the following is the witness (testimony) of John: when the Jews sent priests and Levites from Jerusalem to ask him, “Who are you?” And he promised — and did not lie — and he promised that, “I am not the Christ (lit. the anointed one)”. And they asked him, “Who (are you) then? Are you Elias?”. And he said, “I am not”. “You are the prophet?”, (they asked) and he answered, “No”. They then said, “Who are you? So that we might give an answer to those who sent us. What do you (have to) say about yourself?” And he said, “I am a voice crying out in the wilderness. Straighten the way of the Lord, just as Esaias — the prophet — told you to.” 
		\pend
		\pstart
		And (as) they were sent by the Pharisees, they asked him and said to him, “Why, then, do you baptise (people) if you are neither Christ nor Elias nor the prophet?” And John answered them saying, “I baptise with (using?) water. Amid you there has stood someone whom you did not know — who comes after me and whose sandal straps I am unworthy of untying.” (All of) this happened in Bethany — (a place) opposite the river Jordan — where John was baptising. 
		\pend
		\pstart
		On the next day, he sees Jesus walking toward him and says, “Look! The lamb of God who takes away the sin of the world. It is him about whom I said, ‘After me there goes a man who came before me — for he was before me. And I did not know him; but I came to baptise with water so that he might be revealed to all of Israel.” And John testified saying that, “I have seen the Spirit descending like a dove from the sky (from Heaven) — and it remained upon him. And I did not know him. But he who sent me to baptise in water, that one told me, ‘The one baptising in the Holy Spirit is he upon whom you see the Spirit descending and upon whom you see it remaining. And I saw and testify that he is the Son of God.” 
		\pend
		\pstart
		On the following day, John — and two of his disciples (students) — again stood up and saw Jesus walking about and he said, “Look! The lamb of God.” And his two disciples heard him speak and followed Jesus. Turning around, Jesus saw their following him and said, “What is it you seek?” And they told him, “Rabbi (which means, when translated, teacher), where do you live?” And he told them, “Come and see”. Thus, they went and saw where he lived and stayed with him that day — the time was approximately ten. Andrew — Simon Peter’s brother — was one of the disciples who heard and followed John. He first found his own brother Simon and told him, “We have found the Messiah (which translates to ‘the Anointed One’)”. He led him to Jesus and upon Jesus’ seeing him, Jesus said, “You are Simon, son of John — you shall be called Cephas (which means stone, i. e. Peter).”
		\pend
		\pstart
		On the next day, they wanted to leave to go to Galilee and Jesus finds Philip and tells him, “Follow me.” Philip was from Bethany, from the town of Andrew and Peter. Philip, then, finds Nathanael and tells him, “We have found him of whom Moses wrote in the law — and also the prophets: Jesus, son of Joseph from Nazareth.” And Nathanael told him, “There can be something good from Nazareth?” And Philip told him, “Come and see”. Jesus saw Nathanael coming toward him and said about him, “Look! Truly an Israelite within whom there is no treachery (deceit).” And Nathanael tells (asks) him, “From where do you know me?” And Jesus answered and told him, “Before Phillip called you, I stood (was) beneath the fig tree and saw you.” And Nathanael answered him, “Rabbi, you are the Son of God — you are the king of Israel.” And Jesus answered him saying, “Because I told you that I saw you (whilst I was) below the fig tree, you believe? You shall see greater (things) than these.” And he tells him, “Amen amen, I am telling you: you shall see Heaven open and the angels of God ascending and descending onto the Son of Humankind.”
		\pend
        \endnumbering
    \end{Leftside}

\end{pages} 
\Pages

\clearpage
\thispagestyle{empty}
\null\vfill
\settowidth\longest{\huge\itshape […] and when I turned around I saw}
\begin{center}
\parbox{\longest}{%
  \raggedright{\huge\itshape%
    ``There can be something good from Nazareth?'' \par\bigskip
  }
}
\vfill\vfill
\clearpage\newpage
\end{center}
\newpage
\thispagestyle{empty}
\null\vfill
\settowidth\longest{\huge\itshape […] and when I turned around I saw}
\begin{center}
\parbox{\longest}{%
  \raggedright{\huge\itshape%
    «Ἐκ Ναζαρὲτ δύναταί τι ἀγαθὸν εἶναι;» \par\bigskip
  }
}
\end{center}
\vfill\vfill