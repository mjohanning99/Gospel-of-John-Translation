\begin{pages}
    \begin{Rightside}
    \selectlanguage{greek}
        \beginnumbering
		\pstart[
				\chapter{Ἐν ἀρχῇ ἦν ὁ λόγος}
				\markboth{In the beginning was the Word}
				]
		\renewcommand{\LettrineFontHook}{\PHtitl}
		\lettrine[lines=3]Ἐν ἀρχῇ ἦν ὁ Λόγος, καὶ ὁ Λόγος ἦν πρὸς τὸν Θεόν, καὶ Θεὸς ἦν ὁ Λόγος. Οὗτος ἦν ἐν ἀρχῇ πρὸς τὸν Θεόν. πάντα δι’ αὐτοῦ ἐγένετο, καὶ χωρὶς αὐτοῦ ἐγένετο οὐδὲ ἕν ὃ γέγονεν. ἐν αὐτῷ ζωὴ ἦν, καὶ ἡ ζωὴ ἦν τὸ φῶς τῶν ἀνθρώπων. καὶ τὸ φῶς ἐν τῇ σκοτίᾳ φαίνει, καὶ ἡ σκοτία αὐτὸ οὐ κατέλαβεν. Ἐγένετο ἄνθρωπος, ἀπεσταλμένος παρὰ Θεοῦ, ὄνομα αὐτῷ Ἰωάνης· οὗτος ἦλθεν εἰς μαρτυρίαν, ἵνα μαρτυρήσῃ περὶ τοῦ φωτός, ἵνα πάντες πιστεύσωσιν δι’ αὐτοῦ. οὐκ ἦν ἐκεῖνος τὸ φῶς, ἀλλ’ ἵνα μαρτυρήσῃ περὶ τοῦ φωτός. Ἦν τὸ φῶς τὸ ἀληθινὸν, ὃ φωτίζει πάντα ἄνθρωπον, ἐρχόμενον εἰς τὸν κόσμον. ἐν τῷ κόσμῳ ἦν, καὶ ὁ κόσμος δι’ αὐτοῦ ἐγένετο, καὶ ὁ κόσμος αὐτὸν οὐκ ἔγνω. εἰς τὰ ἴδια ἦλθεν, καὶ οἱ ἴδιοι αὐτὸν οὐ παρέλαβον. ὅσοι δὲ ἔλαβον αὐτόν, ἔδωκεν αὐτοῖς ἐξουσίαν τέκνα Θεοῦ γενέσθαι, τοῖς πιστεύουσιν εἰς τὸ ὄνομα αὐτοῦ, οἳ οὐκ ἐξ αἱμάτων οὐδὲ ἐκ θελήματος σαρκὸς οὐδὲ ἐκ θελήματος ἀνδρὸς ἀλλ’ ἐκ Θεοῦ ἐγεννήθησαν. Καὶ ὁ Λόγος σὰρξ ἐγένετο καὶ ἐσκήνωσεν ἐν ἡμῖν, καὶ ἐθεασάμεθα τὴν δόξαν αὐτοῦ, δόξαν ὡς μονογενοῦς παρὰ Πατρός, πλήρης χάριτος καὶ ἀληθείας. Ἰωάνης μαρτυρεῖ περὶ αὐτοῦ καὶ κέκραγεν λέγων Οὗτος ἦν ὃν εἶπον Ὁ ὀπίσω μου ἐρχόμενος ἔμπροσθέν μου γέγονεν, ὅτι πρῶτός μου ἦν. ὅτι ἐκ τοῦ πληρώματος αὐτοῦ ἡμεῖς πάντες ἐλάβομεν, καὶ χάριν ἀντὶ χάριτος· ὅτι ὁ νόμος διὰ Μωϋσέως ἐδόθη, ἡ χάρις καὶ ἡ ἀλήθεια διὰ Ἰησοῦ Χριστοῦ ἐγένετο. Θεὸν οὐδεὶς ἑώρακεν πώποτε· μονογενὴς Θεὸς ὁ ὢν εἰς τὸν κόλπον τοῦ Πατρὸς, ἐκεῖνος ἐξηγήσατο.
		\pend
		\pstart
		 Καὶ αὕτη ἐστὶν ἡ μαρτυρία τοῦ Ἰωάνου ὅτε ἀπέστειλαν πρὸς αὐτὸν οἱ Ἰουδαῖοι ἐξ Ἱεροσολύμων ἱερεῖς καὶ Λευείτας ἵνα ἐρωτήσωσιν αὐτόν Σὺ τίς εἶ; καὶ ὡμολόγησεν καὶ οὐκ ἠρνήσατο, καὶ ὡμολόγησεν ὅτι Ἐγὼ οὐκ εἰμὶ ὁ Χριστός. καὶ ἠρώτησαν αὐτόν Τί οὖν; σὺ Ἡλείας εἶ; καὶ λέγει Οὐκ εἰμί. Ὁ προφήτης εἶ σύ; καὶ ἀπεκρίθη Οὔ. εἶπαν οὖν αὐτῷ Τίς εἶ; ἵνα ἀπόκρισιν δῶμεν τοῖς πέμψασιν ἡμᾶς· τί λέγεις περὶ σεαυτοῦ; ἔφη Ἐγὼ φωνὴ βοῶντος ἐν τῇ ἐρήμῳ Εὐθύνατε τὴν ὁδὸν Κυρίου, καθὼς εἶπεν Ἡσαΐας ὁ προφήτης. Καὶ ἀπεσταλμένοι ἦσαν ἐκ τῶν Φαρισαίων. καὶ ἠρώτησαν αὐτὸν καὶ εἶπαν αὐτῷ Τί οὖν βαπτίζεις εἰ σὺ οὐκ εἶ ὁ Χριστὸς οὐδὲ Ἡλείας οὐδὲ ὁ προφήτης; ἀπεκρίθη αὐτοῖς ὁ Ἰωάνης λέγων Ἐγὼ βαπτίζω ἐν ὕδατι· μέσος ὑμῶν στήκει ὃν ὑμεῖς οὐκ οἴδατε, ὁ ὀπίσω μου ἐρχόμενος, οὗ οὐκ εἰμὶ ἐγὼ ἄξιος ἵνα λύσω αὐτοῦ τὸν ἱμάντα τοῦ ὑποδήματος. Ταῦτα ἐν Βηθανίᾳ ἐγένετο πέραν τοῦ Ἰορδάνου, ὅπου ἦν ὁ Ἰωάνης βαπτίζων. Τῇ ἐπαύριον βλέπει τὸν Ἰησοῦν ἐρχόμενον πρὸς αὐτόν, καὶ λέγει Ἴδε ὁ Ἀμνὸς τοῦ Θεοῦ ὁ αἴρων τὴν ἁμαρτίαν τοῦ κόσμου. οὗτός ἐστιν ὑπὲρ οὗ ἐγὼ εἶπον Ὀπίσω μου ἔρχεται ἀνὴρ ὃς ἔμπροσθέν μου γέγονεν, ὅτι πρῶτός μου ἦν. κἀγὼ οὐκ ᾔδειν αὐτόν, ἀλλ’ ἵνα φανερωθῇ τῷ Ἰσραὴλ, διὰ τοῦτο ἦλθον ἐγὼ ἐν ὕδατι βαπτίζων. Καὶ ἐμαρτύρησεν Ἰωάνης λέγων ὅτι Τεθέαμαι τὸ Πνεῦμα καταβαῖνον ὡς περιστερὰν ἐξ οὐρανοῦ, καὶ ἔμεινεν ἐπ’ αὐτόν. κἀγὼ οὐκ ᾔδειν αὐτόν, ἀλλ’ ὁ πέμψας με βαπτίζειν ἐν ὕδατι ἐκεῖνός μοι εἶπεν Ἐφ’ ὃν ἂν ἴδῃς τὸ Πνεῦμα καταβαῖνον καὶ μένον ἐπ’ αὐτόν, οὗτός ἐστιν ὁ βαπτίζων ἐν Πνεύματι Ἁγίῳ. κἀγὼ ἑώρακα, καὶ μεμαρτύρηκα ὅτι οὗτός ἐστιν ὁ Υἱὸς τοῦ Θεοῦ.
		\pend
		\pstart		
	 	Τῇ ἐπαύριον πάλιν εἱστήκει ὁ Ἰωάνης καὶ ἐκ τῶν μαθητῶν αὐτοῦ δύο, καὶ ἐμβλέψας τῷ Ἰησοῦ περιπατοῦντι λέγει Ἴδε ὁ Ἀμνὸς τοῦ Θεοῦ. καὶ ἤκουσαν οἱ δύο μαθηταὶ αὐτοῦ λαλοῦντος καὶ ἠκολούθησαν τῷ Ἰησοῦ. στραφεὶς δὲ ὁ Ἰησοῦς καὶ θεασάμενος αὐτοὺς ἀκολουθοῦντας λέγει αὐτοῖς Τί ζητεῖτε; οἱ δὲ εἶπαν αὐτῷ Ῥαββεί, (ὃ λέγεται μεθερμηνευόμενον Διδάσκαλε,) ποῦ μένεις; λέγει αὐτοῖς Ἔρχεσθε καὶ ὄψεσθε. ἦλθαν οὖν καὶ εἶδαν ποῦ μένει, καὶ παρ’ αὐτῷ ἔμειναν τὴν ἡμέραν ἐκείνην· ὥρα ἦν ὡς δεκάτη. Ἦν Ἀνδρέας ὁ ἀδελφὸς Σίμωνος Πέτρου εἷς ἐκ τῶν δύο τῶν ἀκουσάντων παρὰ Ἰωάνου καὶ ἀκολουθησάντων αὐτῷ· εὑρίσκει οὗτος πρῶτον τὸν ἀδελφὸν τὸν ἴδιον Σίμωνα καὶ λέγει αὐτῷ Εὑρήκαμεν τὸν Μεσσίαν (ὅ ἐστιν μεθερμηνευόμενον Χριστός). ἤγαγεν αὐτὸν πρὸς τὸν Ἰησοῦν. ἐμβλέψας αὐτῷ ὁ Ἰησοῦς εἶπεν Σὺ εἶ Σίμων ὁ υἱὸς Ἰωάνου, σὺ κληθήσῃ Κηφᾶς (ὃ ἑρμηνεύεται Πέτρος). Τῇ ἐπαύριον ἠθέλησεν ἐξελθεῖν εἰς τὴν Γαλιλαίαν. καὶ εὑρίσκει Φίλιππον. καὶ λέγει αὐτῷ ὁ Ἰησοῦς Ἀκολούθει μοι. ἦν δὲ ὁ Φίλιππος ἀπὸ Βηθσαϊδά, ἐκ τῆς πόλεως Ἀνδρέου καὶ Πέτρου. εὑρίσκει Φίλιππος τὸν Ναθαναὴλ καὶ λέγει αὐτῷ Ὃν ἔγραψεν Μωϋσῆς ἐν τῷ νόμῳ καὶ οἱ προφῆται εὑρήκαμεν, Ἰησοῦν υἱὸν τοῦ Ἰωσὴφ τὸν ἀπὸ Ναζαρέτ. καὶ εἶπεν αὐτῷ Ναθαναήλ Ἐκ Ναζαρὲτ δύναταί τι ἀγαθὸν εἶναι; λέγει αὐτῷ ὁ Φίλιππος Ἔρχου καὶ ἴδε. εἶδεν Ἰησοῦς τὸν Ναθαναὴλ ἐρχόμενον πρὸς αὐτὸν καὶ λέγει περὶ αὐτοῦ Ἴδε ἀληθῶς Ἰσραηλείτης, ἐν ᾧ δόλος οὐκ ἔστιν. λέγει αὐτῷ Ναθαναήλ Πόθεν με γινώσκεις; ἀπεκρίθη Ἰησοῦς καὶ εἶπεν αὐτῷ Πρὸ τοῦ σε Φίλιππον φωνῆσαι ὄντα ὑπὸ τὴν συκῆν εἶδόν σε. ἀπεκρίθη αὐτῷ Ναθαναήλ Ῥαββεί, σὺ εἶ ὁ Υἱὸς τοῦ Θεοῦ, σὺ Βασιλεὺς εἶ τοῦ Ἰσραήλ. ἀπεκρίθη Ἰησοῦς καὶ εἶπεν αὐτῷ Ὅτι εἶπόν σοι ὅτι εἶδόν σε ὑποκάτω τῆς συκῆς, πιστεύεις; μείζω τούτων ὄψῃ. καὶ λέγει αὐτῷ Ἀμὴν ἀμὴν λέγω ὑμῖν, ὄψεσθε τὸν οὐρανὸν ἀνεῳγότα καὶ τοὺς ἀγγέλους τοῦ Θεοῦ ἀναβαίνοντας καὶ καταβαίνοντας ἐπὶ τὸν Υἱὸν τοῦ ἀνθρώπου.
		\pend
        \endnumbering
    \end{Rightside}
    \begin{Leftside}
        \beginnumbering
        \pstart[
        			\chapter{In the Beginning Was the Word}
        			]
        		\renewcommand\LettrineFontHook{\Zallmanfamily}
			\lettrine[lines=3]{T}{he} Revelation of Jesus Christ, which God gave Him to show His servants what must soon happen; and He made it known through the sending of His messenger to His servant John, who confirms everything that he saw, namely the word of God and the testimony of Jesus Christ. Blessed is the reader and the people who listen to the words of the prophecy and (blessed is) the one who heeds what is written in it (the prophecy), for the time is near.
		\pend
		\pstart
		\pend
		\pstart
		\pend
        \endnumbering
    \end{Leftside}

\end{pages} 
\Pages

\clearpage
\thispagestyle{empty}
\null\vfill
\settowidth\longest{\huge\itshape […] and when I turned around I saw}
\begin{center}
\parbox{\longest}{%
  \raggedright{\huge\itshape%
    ``[…] and when I turned around I saw seven golden lamp-stands; and in the midst of the lamp-stands was someone like the Son of Man.'' \par\bigskip
  }
  \raggedleft\Large\MakeUppercase{``Menschensohn'' — Gebhard Fugel, 1933}\par%
}
\vfill\vfill
\clearpage\newpage
\end{center}
\newpage
\thispagestyle{empty}
\begin{center}
	%\includegraphics[width=0.98\textwidth]{}
\end{center}