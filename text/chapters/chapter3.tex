\begin{pages}
    \begin{Rightside}
    \selectlanguage{greek}
        \beginnumbering
		\pstart[
				\chapter{Τὸ ἄνωθεν γεννηθῆναι}
				\markboth{To be born anew}
				]
		\renewcommand{\LettrineFontHook}{\PHtitl}
		\lettrine[lines=3]{Ἦ}{ν} δὲ ἄνθρωπος ἐκ τῶν Φαρισαίων, Νικόδημος ὄνομα αὐτῷ, ἄρχων τῶν Ἰουδαίων· οὗτος ἦλθεν πρὸς αὐτὸν νυκτὸς καὶ εἶπεν αὐτῷ Ῥαββεί, οἴδαμεν ὅτι ἀπὸ Θεοῦ ἐλήλυθας διδάσκαλος· οὐδεὶς γὰρ δύναται ταῦτα τὰ σημεῖα ποιεῖν ἃ σὺ ποιεῖς, ἐὰν μὴ ᾖ ὁ Θεὸς μετ’ αὐτοῦ. ἀπεκρίθη Ἰησοῦς καὶ εἶπεν αὐτῷ Ἀμὴν ἀμὴν λέγω σοι, ἐὰν μή τις γεννηθῇ ἄνωθεν, οὐ δύναται ἰδεῖν τὴν βασιλείαν τοῦ Θεοῦ. λέγει πρὸς αὐτὸν ὁ Νικόδημος Πῶς δύναται ἄνθρωπος γεννηθῆναι γέρων ὤν; μὴ δύναται εἰς τὴν κοιλίαν τῆς μητρὸς αὐτοῦ δεύτερον εἰσελθεῖν καὶ γεννηθῆναι; ἀπεκρίθη Ἰησοῦς Ἀμὴν ἀμὴν λέγω σοι, ἐὰν μή τις γεννηθῇ ἐξ ὕδατος καὶ Πνεύματος, οὐ δύναται εἰσελθεῖν εἰς τὴν βασιλείαν τοῦ Θεοῦ. τὸ γεγεννημένον ἐκ τῆς σαρκὸς σάρξ ἐστιν, καὶ τὸ γεγεννημένον ἐκ τοῦ Πνεύματος πνεῦμά ἐστιν. μὴ θαυμάσῃς ὅτι εἶπόν σοι Δεῖ ὑμᾶς γεννηθῆναι ἄνωθεν. τὸ πνεῦμα ὅπου θέλει πνεῖ, καὶ τὴν φωνὴν αὐτοῦ ἀκούεις, ἀλλ’ οὐκ οἶδας πόθεν ἔρχεται καὶ ποῦ ὑπάγει· οὕτως ἐστὶν πᾶς ὁ γεγεννημένος ἐκ τοῦ Πνεύματος. 
		\pend
		\pstart
		Ἀπεκρίθη Νικόδημος καὶ εἶπεν αὐτῷ Πῶς δύναται ταῦτα γενέσθαι; ἀπεκρίθη Ἰησοῦς καὶ εἶπεν αὐτῷ Σὺ εἶ ὁ διδάσκαλος τοῦ Ἰσραὴλ καὶ ταῦτα οὐ γινώσκεις; ἀμὴν ἀμὴν λέγω σοι ὅτι ὃ οἴδαμεν λαλοῦμεν καὶ ὃ ἑωράκαμεν μαρτυροῦμεν, καὶ τὴν μαρτυρίαν ἡμῶν οὐ λαμβάνετε. εἰ τὰ ἐπίγεια εἶπον ὑμῖν καὶ οὐ πιστεύετε, πῶς ἐὰν εἴπω ὑμῖν τὰ ἐπουράνια πιστεύσετε; καὶ οὐδεὶς ἀναβέβηκεν εἰς τὸν οὐρανὸν εἰ μὴ ὁ ἐκ τοῦ οὐρανοῦ καταβάς, ὁ Υἱὸς τοῦ ἀνθρώπου. καὶ καθὼς Μωϋσῆς ὕψωσεν τὸν ὄφιν ἐν τῇ ἐρήμῳ, οὕτως ὑψωθῆναι δεῖ τὸν Υἱὸν τοῦ ἀνθρώπου, ἵνα πᾶς ὁ πιστεύων ἐν αὐτῷ ἔχῃ ζωὴν αἰώνιον. 
		\pend
		\pstart
		Οὕτως γὰρ ἠγάπησεν ὁ Θεὸς τὸν κόσμον, ὥστε τὸν Υἱὸν τὸν μονογενῆ ἔδωκεν, ἵνα πᾶς ὁ πιστεύων εἰς αὐτὸν μὴ ἀπόληται ἀλλ’ ἔχῃ ζωὴν αἰώνιον. οὐ γὰρ ἀπέστειλεν ὁ Θεὸς τὸν Υἱὸν εἰς τὸν κόσμον ἵνα κρίνῃ τὸν κόσμον, ἀλλ’ ἵνα σωθῇ ὁ κόσμος δι’ αὐτοῦ. ὁ πιστεύων εἰς αὐτὸν οὐ κρίνεται· ὁ μὴ πιστεύων ἤδη κέκριται, ὅτι μὴ πεπίστευκεν εἰς τὸ ὄνομα τοῦ μονογενοῦς Υἱοῦ τοῦ Θεοῦ. αὕτη δέ ἐστιν ἡ κρίσις, ὅτι τὸ φῶς ἐλήλυθεν εἰς τὸν κόσμον καὶ ἠγάπησαν οἱ ἄνθρωποι μᾶλλον τὸ σκότος ἢ τὸ φῶς· ἦν γὰρ αὐτῶν πονηρὰ τὰ ἔργα. πᾶς γὰρ ὁ φαῦλα πράσσων μισεῖ τὸ φῶς καὶ οὐκ ἔρχεται πρὸς τὸ φῶς, ἵνα μὴ ἐλεγχθῇ τὰ ἔργα αὐτοῦ· ὁ δὲ ποιῶν τὴν ἀλήθειαν ἔρχεται πρὸς τὸ φῶς, ἵνα φανερωθῇ αὐτοῦ τὰ ἔργα ὅτι ἐν Θεῷ ἐστιν εἰργασμένα. 
		\pend
		\pstart
		Μετὰ ταῦτα ἦλθεν ὁ Ἰησοῦς καὶ οἱ μαθηταὶ αὐτοῦ εἰς τὴν Ἰουδαίαν γῆν, καὶ ἐκεῖ διέτριβεν μετ’ αὐτῶν καὶ ἐβάπτιζεν. ἦν δὲ καὶ Ἰωάνης βαπτίζων ἐν Αἰνὼν ἐγγὺς τοῦ Σαλείμ, ὅτι ὕδατα πολλὰ ἦν ἐκεῖ, καὶ παρεγίνοντο καὶ ἐβαπτίζοντο· οὔπω γὰρ ἦν βεβλημένος εἰς τὴν φυλακὴν Ἰωάνης. 
		\pend
		\pstart
		Ἐγένετο οὖν ζήτησις ἐκ τῶν μαθητῶν Ἰωάνου μετὰ Ἰουδαίου περὶ καθαρισμοῦ. καὶ ἦλθον πρὸς τὸν Ἰωάνην καὶ εἶπαν αὐτῷ Ῥαββεί, ὃς ἦν μετὰ σοῦ πέραν τοῦ Ἰορδάνου, ᾧ σὺ μεμαρτύρηκας, ἴδε οὗτος βαπτίζει καὶ πάντες ἔρχονται πρὸς αὐτόν. ἀπεκρίθη Ἰωάνης καὶ εἶπεν Οὐ δύναται ἄνθρωπος λαμβάνειν οὐδὲν ἐὰν μὴ ᾖ δεδομένον αὐτῷ ἐκ τοῦ οὐρανοῦ. αὐτοὶ ὑμεῖς μοι μαρτυρεῖτε ὅτι εἶπον Οὐκ εἰμὶ ἐγὼ ὁ Χριστός, ἀλλ’ ὅτι Ἀπεσταλμένος εἰμὶ ἔμπροσθεν ἐκείνου. Ὁ ἔχων τὴν νύμφην νυμφίος ἐστίν· ὁ δὲ φίλος τοῦ νυμφίου ὁ ἑστηκὼς καὶ ἀκούων αὐτοῦ, χαρᾷ χαίρει διὰ τὴν φωνὴν τοῦ νυμφίου. αὕτη οὖν ἡ χαρὰ ἡ ἐμὴ πεπλήρωται. ἐκεῖνον δεῖ αὐξάνειν, ἐμὲ δὲ ἐλαττοῦσθαι. 
		\pend
		\pstart
		Ὁ ἄνωθεν ἐρχόμενος ἐπάνω πάντων ἐστίν· ὁ ὢν ἐκ τῆς γῆς ἐκ τῆς γῆς ἐστιν καὶ ἐκ τῆς γῆς λαλεῖ. ὁ ἐκ τοῦ οὐρανοῦ ἐρχόμενος ἐπάνω πάντων ἐστίν· ὃ ἑώρακεν καὶ ἤκουσεν, τοῦτο μαρτυρεῖ, καὶ τὴν μαρτυρίαν αὐτοῦ οὐδεὶς λαμβάνει. ὁ λαβὼν αὐτοῦ τὴν μαρτυρίαν ἐσφράγισεν ὅτι ὁ Θεὸς ἀληθής ἐστιν. ὃν γὰρ ἀπέστειλεν ὁ Θεὸς τὰ ῥήματα τοῦ Θεοῦ λαλεῖ· οὐ γὰρ ἐκ μέτρου δίδωσιν τὸ Πνεῦμα. ὁ Πατὴρ ἀγαπᾷ τὸν Υἱόν, καὶ πάντα δέδωκεν ἐν τῇ χειρὶ αὐτοῦ. ὁ πιστεύων εἰς τὸν Υἱὸν ἔχει ζωὴν αἰώνιον· ὁ δὲ ἀπειθῶν τῷ Υἱῷ οὐκ ὄψεται ζωήν, ἀλλ’ ἡ ὀργὴ τοῦ Θεοῦ μένει ἐπ’ αὐτόν.
		\pend
        \endnumbering
    \end{Rightside}
    \begin{Leftside}
        \beginnumbering
        \pstart[
        			\chapter{To Be Born Anew}
        			]
        		\renewcommand\LettrineFontHook{\Zallmanfamily}
		\lettrine[lines=3]{T}{here} was a man of the Pharisees — Nicodemus was his name — (and he was) a ruler of the Jews. He went to him (to Jesus) one night and told him, “Rabbi, we know that you have come from God (as) a teacher; for nobody would be able to perform the signs (miracles) that you perform unless God was with them.” Jesus answered and told him, “Amen amen, I tell you: unless someone is born again (unless someone is born from above), they are not able to see the Kingdom of God.” Nicodemus says to him (in response), “How can someone be born if they are old? Surely they cannot enter their mother’s womb a second time and be born (again)?” Jesus answered, “Amen amen, I tell you: unless someone is born from (of) water and the Spirit, they cannot enter the Kingdom of God. That (which was) born of flesh is flesh; and that (which was) born out of the Spirit is the Spirit. Do not be amazed that I told you, ‘We must be born anew (from above)’. The wind (Spirit) blows where it wishes — and you hear its voice but do not know from where it is coming and where it is going. Such is everything (everyone) (that was) born of the Spirit.” 
		\pend
		\pstart
		Nicodemus answered and told him, “How is it possible for this to happen (how can this happen)?” And Jesus answered and told him, “You are the teacher of Israel and you do not know this? Amen amen, I tell you that (those things) we have known, we speak; and that which we have seen, we bear witness to — and (yet) you do not accept (receive, take) our testimony. If, when I tell you about earthly things, you do not believe — how will you believe (me) when I tell you about heavenly things? And nobody has ascended into Heaven except for him who came out Heaven — the Son of Man. And just as Moses lifted up the serpent in the wilderness, the Son of Man must, too, be lifted up so that everyone who believes in him might have eternal life. 
		\pend
		\pstart
		“For God loved the world (cosmos) so (much) that he gave his one and only Son in order that everyone who believes in him shall not perish, but shall (instead) have eternal life. For God did not send his Son into the world (cosmos) so that he (his Son) might judge the world; but so that the world might be saved through him. Those who believe in him are not judged and those who do not believe have already been judged, for they did not believe in the name of the one and only Son of God. And such is the judgement: that the light has come into the world (cosmos) and the humans loved (liked) the darkness more than the light; because their deeds were evil. For everyone who does evil (bad, worthless) things hates the light and will not walk (come) toward the light so that his deeds might not be exposed. (But) those who practice (do) the truth(ful things), they will go (come) toward the light, so that their deeds shall be revealed; for they (their deeds) are accomplished in (before?) God.”
		\pend
		\pstart
		Hereafter, Jesus and his disciples went into (the) Judean land (territory) and there he spent time with them and baptised (people). And even John was baptising in Ænon — (a place) close to Salim —, for there was a lot of water there. And they came (arrived) and were baptised — since John had not yet been thrown into prison. 
		\pend
		\pstart
		And (then) a dispute (argument) took place between some of John’s disciples and the Jews regarding the ceremonial cleansing (purifying). And they went to John and told him, “Rabbi, he who was with you on the other side of the Jordan (river) and of whom you bore witness — look! He (also) baptises and everyone comes to him.” John answered and said, “A person cannot receive (take) anything unless it was given to him from Heaven. You yourselves (can) testify to my saying that, ‘I am not the Christ but I am sent before him’. He who has a bride is the bridegroom; but the one who is the bridegroom’s friend — who stands and listens to him —, they rejoice by hearing the bridegroom’s voice. My joy has (thus?) been fulfilled (completed). He (over there) must grow (be increased) and I must be lessened (decreased).
		\pend
		\pstart
		“Who comes from above is above all and who is of the Earth is of the Earth and speaks of the Earth; who comes from Heaven is above all. And what he has seen and heard, to that he bears witness and nobody takes his testimony. And the taker (receiver) of the testimony seals (acknowledges?) that God is true. For he whom God sent speaks the words of God; for God does not give the Spirit by measure. The Father loves the Son and he placed everything into his hand. Who believes in the Son has eternal life; but who disobeys the Son shall not see life, but God’s wrath shall rest upon them.” 
		\pend
        \endnumbering
    \end{Leftside}
\end{pages} 
\Pages

\clearpage
\thispagestyle{empty}
\null\vfill
\settowidth\longest{\huge\itshape […] and when I turned around I saw}
\begin{center}
\parbox{\longest}{%
  \raggedright{\huge\itshape%
    ``How can someone be born if they are old? Surely they cannot enter their mother’s womb a second time and be born (again)?'' \par\bigskip
  }
}
\vfill\vfill
\clearpage\newpage
\end{center}
\newpage
\thispagestyle{empty}
\null\vfill
\settowidth\longest{\huge\itshape […] and when I turned around I saw}
\begin{center}
\parbox{\longest}{%
  \raggedright{\huge\itshape%
    «Πῶς δύναται ἄνθρωπος γεννηθῆναι γέρων ὤν; μὴ δύναται εἰς τὴν κοιλίαν τῆς μητρὸς αὐτοῦ δεύτερον εἰσελθεῖν καὶ γεννηθῆναι;» \par\bigskip
  }
}
\end{center}
\vfill\vfill