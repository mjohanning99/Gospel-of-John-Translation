\chapter*{Author's Note  \\ \large On obtaining a copy, compilation and license information}
	\markboth{Author's Note}{Author's Note}
	\addcontentsline{toc}{chapter}{Author's Note}

As this book is not “copyrighted” in the traditional sense, I thought it prudent to write a short informational text here, explaining how you may legally attain your own (free) copy of my work, how you may use it and how to support me. 

First and foremost, it is important to note that I am not making a non-watermarked PDF file of this book freely available; you may, however, at any time, download its source-code and compile it into any format that you wish — this is made possible through my having written this book in \LaTeX. This compilation process requires a full \LaTeX\ installation and a \LaTeX\ compiler compatible with my book, preferably \XeLaTeX\ or Lua\LaTeX; I would, however, strongly advise against the usage of pdf\LaTeX, as the Greek text appears to trouble it greatly and prevents it from working properly — or indeed, at all. If you wish to receive more information regarding the installation of a \TeX\ environment on your particular system and the compilation of documents, please refer to the official \LaTeX\ website (\url{https://www.latex-project.org/}) — installation is, generally, pretty straight-forward (at least on the systems that I use, i. e. macOS Big Sur and various Linux distributions). 

An important thing to note would be the fact that the root document file — from which all other parts of the document, such as the various chapters and the front matter, are loaded — is titled \textit{gospelofjohn.tex}; compiling the book can, therefore, be accomplished by simply typing “xelatex gospelofjohn.tex” (or the equivalent command for another compiler) in the project's main folder. 

After compilation, you may use the book — and your compiled document — in accordance with the Creative Commons BY-NC-ND 4.0 International license; feel free to share your compiled version with your colleagues and friends, but do not attempt to sell it without my prior approval. Additionally, should you wish to use my work for a purpose that is generally forbidden by the license, I encourage you to contact me — I am fairly certain we can come to an agreement.

If you wish to obtain an official copy of my book — either digitally or in print —, then I highly encourage you to check out my website for always up-to-date information regarding the availability of various editions; you can find this, and some additional information, by following the following link: \url{https://ancient-greek.net/books/gospelofjohn.php}. This webpage also contains downloads for both a watermarked PDF preview and the archived (usually a regular ZIP file) \LaTeX\ source code of the book; furthermore, you may find the source code on this book’s GitHub repository: \url{https://github.com/mjohanning99/Gospel-of-John-Translation}.

At any rate, however, I hope that you enjoy my translation; and should you find things that could be improved or that need to be correct, I would love to hear from you! You can find my email at the beginning of the book. 